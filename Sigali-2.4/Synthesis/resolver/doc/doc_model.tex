% -*- Latex -*-
\chapter{Les mod�les}
\section{Mod�les des signaux}
Le language \signal utilis� dans cet outil d'aide � la conception d'automatismes
logiques, d�crit des ensembles de signaux. Un signal est une suite infinie d'�l�ments
d'un domaine qui peut �tre les bool�ens, les entiers, les r�els ou tout type construit
� l'aide de constructeurs fournis par le langage. Dans ce mod�le, l'indice $n$ d'un 
�l�ment $x_n$ d'une suite $X=(x_i)_{i\in N}$ indique simplement un num�ro d'ordre dans
la suite mais ne doit, en aucun cas, �tre interpr�t�, a priori, 
 comme un top d'une horloge. Une
horloge pourra d'ailleur �tre mod�lis�e comme une suite $H$ de $top$ par exemple. 

Maintenant que nous disposons d'un signal $X$ que nous pouvons supposer � valeurs 
num�riques pour fixer les id�es, et d'un autre signal � valeurs dans $\{ top \}$, nous
aimerions dire que le second est l'horloge d'�mission du premier. Ceci signifie tout
simplement que les valeurs des signaux $X$ et $H$ apparaissent simultan�ment et plus
pr�cis�ment la nieme valeur de $X$ et la nieme valeur de $H$ apparaissent simultan�ment.

Si au contraire nous ne connaissons aucune relation entre les apparitions de $X$ et
de $H$, nous devons admettre qu'un nombre fini quelconque de valeurs de $X$ peuvent
�tre �mises entre les apparitions de deux valeurs successives de $H$ et vice-versa.
Bien entendu, nous admettons �galement la possibilit� d'�missions simultan� de valeurs
pour les deux signaux. 